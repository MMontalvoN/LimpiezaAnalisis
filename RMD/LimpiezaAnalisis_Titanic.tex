% Options for packages loaded elsewhere
\PassOptionsToPackage{unicode}{hyperref}
\PassOptionsToPackage{hyphens}{url}
%
\documentclass[
]{article}
\title{Tipología y ciclo de vida de los datos}
\author{Miguel Ángel Montalvo Navidad}
\date{Enero 2022}

\usepackage{amsmath,amssymb}
\usepackage{lmodern}
\usepackage{iftex}
\ifPDFTeX
  \usepackage[T1]{fontenc}
  \usepackage[utf8]{inputenc}
  \usepackage{textcomp} % provide euro and other symbols
\else % if luatex or xetex
  \usepackage{unicode-math}
  \defaultfontfeatures{Scale=MatchLowercase}
  \defaultfontfeatures[\rmfamily]{Ligatures=TeX,Scale=1}
\fi
% Use upquote if available, for straight quotes in verbatim environments
\IfFileExists{upquote.sty}{\usepackage{upquote}}{}
\IfFileExists{microtype.sty}{% use microtype if available
  \usepackage[]{microtype}
  \UseMicrotypeSet[protrusion]{basicmath} % disable protrusion for tt fonts
}{}
\makeatletter
\@ifundefined{KOMAClassName}{% if non-KOMA class
  \IfFileExists{parskip.sty}{%
    \usepackage{parskip}
  }{% else
    \setlength{\parindent}{0pt}
    \setlength{\parskip}{6pt plus 2pt minus 1pt}}
}{% if KOMA class
  \KOMAoptions{parskip=half}}
\makeatother
\usepackage{xcolor}
\IfFileExists{xurl.sty}{\usepackage{xurl}}{} % add URL line breaks if available
\IfFileExists{bookmark.sty}{\usepackage{bookmark}}{\usepackage{hyperref}}
\hypersetup{
  pdftitle={Tipología y ciclo de vida de los datos},
  pdfauthor={Miguel Ángel Montalvo Navidad},
  hidelinks,
  pdfcreator={LaTeX via pandoc}}
\urlstyle{same} % disable monospaced font for URLs
\usepackage[margin=1in]{geometry}
\usepackage{color}
\usepackage{fancyvrb}
\newcommand{\VerbBar}{|}
\newcommand{\VERB}{\Verb[commandchars=\\\{\}]}
\DefineVerbatimEnvironment{Highlighting}{Verbatim}{commandchars=\\\{\}}
% Add ',fontsize=\small' for more characters per line
\usepackage{framed}
\definecolor{shadecolor}{RGB}{48,48,48}
\newenvironment{Shaded}{\begin{snugshade}}{\end{snugshade}}
\newcommand{\AlertTok}[1]{\textcolor[rgb]{1.00,0.81,0.69}{#1}}
\newcommand{\AnnotationTok}[1]{\textcolor[rgb]{0.50,0.62,0.50}{\textbf{#1}}}
\newcommand{\AttributeTok}[1]{\textcolor[rgb]{0.80,0.80,0.80}{#1}}
\newcommand{\BaseNTok}[1]{\textcolor[rgb]{0.86,0.64,0.64}{#1}}
\newcommand{\BuiltInTok}[1]{\textcolor[rgb]{0.80,0.80,0.80}{#1}}
\newcommand{\CharTok}[1]{\textcolor[rgb]{0.86,0.64,0.64}{#1}}
\newcommand{\CommentTok}[1]{\textcolor[rgb]{0.50,0.62,0.50}{#1}}
\newcommand{\CommentVarTok}[1]{\textcolor[rgb]{0.50,0.62,0.50}{\textbf{#1}}}
\newcommand{\ConstantTok}[1]{\textcolor[rgb]{0.86,0.64,0.64}{\textbf{#1}}}
\newcommand{\ControlFlowTok}[1]{\textcolor[rgb]{0.94,0.87,0.69}{#1}}
\newcommand{\DataTypeTok}[1]{\textcolor[rgb]{0.87,0.87,0.75}{#1}}
\newcommand{\DecValTok}[1]{\textcolor[rgb]{0.86,0.86,0.80}{#1}}
\newcommand{\DocumentationTok}[1]{\textcolor[rgb]{0.50,0.62,0.50}{#1}}
\newcommand{\ErrorTok}[1]{\textcolor[rgb]{0.76,0.75,0.62}{#1}}
\newcommand{\ExtensionTok}[1]{\textcolor[rgb]{0.80,0.80,0.80}{#1}}
\newcommand{\FloatTok}[1]{\textcolor[rgb]{0.75,0.75,0.82}{#1}}
\newcommand{\FunctionTok}[1]{\textcolor[rgb]{0.94,0.94,0.56}{#1}}
\newcommand{\ImportTok}[1]{\textcolor[rgb]{0.80,0.80,0.80}{#1}}
\newcommand{\InformationTok}[1]{\textcolor[rgb]{0.50,0.62,0.50}{\textbf{#1}}}
\newcommand{\KeywordTok}[1]{\textcolor[rgb]{0.94,0.87,0.69}{#1}}
\newcommand{\NormalTok}[1]{\textcolor[rgb]{0.80,0.80,0.80}{#1}}
\newcommand{\OperatorTok}[1]{\textcolor[rgb]{0.94,0.94,0.82}{#1}}
\newcommand{\OtherTok}[1]{\textcolor[rgb]{0.94,0.94,0.56}{#1}}
\newcommand{\PreprocessorTok}[1]{\textcolor[rgb]{1.00,0.81,0.69}{\textbf{#1}}}
\newcommand{\RegionMarkerTok}[1]{\textcolor[rgb]{0.80,0.80,0.80}{#1}}
\newcommand{\SpecialCharTok}[1]{\textcolor[rgb]{0.86,0.64,0.64}{#1}}
\newcommand{\SpecialStringTok}[1]{\textcolor[rgb]{0.80,0.58,0.58}{#1}}
\newcommand{\StringTok}[1]{\textcolor[rgb]{0.80,0.58,0.58}{#1}}
\newcommand{\VariableTok}[1]{\textcolor[rgb]{0.80,0.80,0.80}{#1}}
\newcommand{\VerbatimStringTok}[1]{\textcolor[rgb]{0.80,0.58,0.58}{#1}}
\newcommand{\WarningTok}[1]{\textcolor[rgb]{0.50,0.62,0.50}{\textbf{#1}}}
\usepackage{graphicx}
\makeatletter
\def\maxwidth{\ifdim\Gin@nat@width>\linewidth\linewidth\else\Gin@nat@width\fi}
\def\maxheight{\ifdim\Gin@nat@height>\textheight\textheight\else\Gin@nat@height\fi}
\makeatother
% Scale images if necessary, so that they will not overflow the page
% margins by default, and it is still possible to overwrite the defaults
% using explicit options in \includegraphics[width, height, ...]{}
\setkeys{Gin}{width=\maxwidth,height=\maxheight,keepaspectratio}
% Set default figure placement to htbp
\makeatletter
\def\fps@figure{htbp}
\makeatother
\setlength{\emergencystretch}{3em} % prevent overfull lines
\providecommand{\tightlist}{%
  \setlength{\itemsep}{0pt}\setlength{\parskip}{0pt}}
\setcounter{secnumdepth}{-\maxdimen} % remove section numbering
\ifLuaTeX
  \usepackage{selnolig}  % disable illegal ligatures
\fi

\begin{document}
\maketitle

{
\setcounter{tocdepth}{2}
\tableofcontents
}
\begin{center}\rule{0.5\linewidth}{0.5pt}\end{center}

\hypertarget{resoluciuxf3n}{%
\section{Resolución}\label{resoluciuxf3n}}

\begin{center}\rule{0.5\linewidth}{0.5pt}\end{center}

\hypertarget{descripciuxf3n-del-dataset}{%
\subsection{Descripción del dataset}\label{descripciuxf3n-del-dataset}}

A continuación, utilizaremos el juego de datos ``Titanic.csv'' que
recoge datos sobre el famoso transatlántico de pasajeros británico.

\hypertarget{importancia-y-objetivos-de-los-anuxe1lisis}{%
\subsection{Importancia y objetivos de los
análisis}\label{importancia-y-objetivos-de-los-anuxe1lisis}}

\begin{itemize}
\tightlist
\item
  Las actividades que llevaremos a cabo en el desarrollo de la siguiente
  práctica hace referenica a la limpieza y análisis de los datos para un
  proyecto de datos. Tiene como objetivo obtener un dominio de los datos
  para su posterior análisis. Tenemos que conocer profundamente los
  datos tanto en su formato como contenido. Tareas típicas pueden ser la
  selección de características o variables, la preparación del juego de
  datos para posteriormente ser consumido por un algoritmo e intentar
  extraer el máximo conocimiento posible de los datos.
\end{itemize}

\hypertarget{limpieza-de-los-datos}{%
\subsection{Limpieza de los datos}\label{limpieza-de-los-datos}}

\begin{itemize}
\tightlist
\item
  Como paso previo procedemos a instalamos y cargar las librerías
  ggplot2 y dplry.
\end{itemize}

\begin{Shaded}
\begin{Highlighting}[]
\CommentTok{\# https://cran.r{-}project.org/web/packages/ggplot2/index.html}
\ControlFlowTok{if}\NormalTok{ (}\SpecialCharTok{!}\FunctionTok{require}\NormalTok{(}\StringTok{\textquotesingle{}ggplot2\textquotesingle{}}\NormalTok{)) }\FunctionTok{install.packages}\NormalTok{(}\StringTok{\textquotesingle{}ggplot2\textquotesingle{}}\NormalTok{); }\FunctionTok{library}\NormalTok{(}\StringTok{\textquotesingle{}ggplot2\textquotesingle{}}\NormalTok{)}
\CommentTok{\# https://cran.r{-}project.org/web/packages/dplyr/index.html}
\ControlFlowTok{if}\NormalTok{ (}\SpecialCharTok{!}\FunctionTok{require}\NormalTok{(}\StringTok{\textquotesingle{}dplyr\textquotesingle{}}\NormalTok{)) }\FunctionTok{install.packages}\NormalTok{(}\StringTok{\textquotesingle{}dplyr\textquotesingle{}}\NormalTok{); }\FunctionTok{library}\NormalTok{(}\StringTok{\textquotesingle{}dplyr\textquotesingle{}}\NormalTok{)}
\end{Highlighting}
\end{Shaded}

\begin{itemize}
\tightlist
\item
  Ahora cargaremos el fichero de datos.
\end{itemize}

\begin{Shaded}
\begin{Highlighting}[]
\NormalTok{totalData }\OtherTok{\textless{}{-}} \FunctionTok{read.csv}\NormalTok{(}\StringTok{\textquotesingle{}titanic.csv\textquotesingle{}}\NormalTok{,}\AttributeTok{stringsAsFactors =} \ConstantTok{FALSE}\NormalTok{)}
\NormalTok{filas}\OtherTok{=}\FunctionTok{dim}\NormalTok{(totalData)[}\DecValTok{1}\NormalTok{]}
\end{Highlighting}
\end{Shaded}

\begin{itemize}
\tightlist
\item
  Procedemos a guardar los datos filtrados por tripulación ``engineering
  crew'' para hacer análisis posteriores.
\end{itemize}

\begin{Shaded}
\begin{Highlighting}[]
\NormalTok{totalData\_crew}\OtherTok{=}\FunctionTok{subset}\NormalTok{(totalData, totalData}\SpecialCharTok{$}\NormalTok{class}\SpecialCharTok{==}\StringTok{"engineering crew"}\NormalTok{)}
\end{Highlighting}
\end{Shaded}

\begin{itemize}
\tightlist
\item
  Con el siguiente comando verificamos la estructura del data set
  principal.
\end{itemize}

\begin{Shaded}
\begin{Highlighting}[]
\FunctionTok{str}\NormalTok{(totalData)}
\end{Highlighting}
\end{Shaded}

\begin{verbatim}
## 'data.frame':    2207 obs. of  11 variables:
##  $ name    : chr  "Abbing, Mr. Anthony" "Abbott, Mr. Eugene Joseph" "Abbott, Mr. Rossmore Edward" "Abbott, Mrs. Rhoda Mary 'Rosa'" ...
##  $ gender  : chr  "male" "male" "male" "female" ...
##  $ age     : num  42 13 16 39 16 25 30 28 27 20 ...
##  $ class   : chr  "3rd" "3rd" "3rd" "3rd" ...
##  $ embarked: chr  "S" "S" "S" "S" ...
##  $ country : chr  "United States" "United States" "United States" "England" ...
##  $ ticketno: int  5547 2673 2673 2673 348125 348122 3381 3381 2699 3101284 ...
##  $ fare    : num  7.11 20.05 20.05 20.05 7.13 ...
##  $ sibsp   : int  0 0 1 1 0 0 1 1 0 0 ...
##  $ parch   : int  0 2 1 1 0 0 0 0 0 0 ...
##  $ survived: chr  "no" "no" "no" "yes" ...
\end{verbatim}

Podemos observar que tenemos 2207 registros que se corresponden a los
viajeros y tripulación del crucero y 11 variables que los caracterizan.

Revisamos la descripción de las variables contenidas al fichero y si los
tipos de variable se corresponde al que hemos cargado:

\textbf{name} string with the name of the passenger.

\textbf{gender} factor with levels male and female.

\textbf{age} numeric value with the persons age on the day of the
sinking. The age of babies (under 12 months) is given as a fraction of
one year (1/month).

\textbf{class} factor specifying the class for passengers or the type of
service aboard for crew members.

\textbf{embarked} factor with the persons place of of embarkment.

\textbf{country} factor with the persons home country.

\textbf{ticketno} numeric value specifying the persons ticket number (NA
for crew members).

\textbf{fare} numeric value with the ticket price (NA for crew members,
musicians and employees of the shipyard company).

\textbf{sibsp} ordered factor specifying the number if siblings/spouses
aboard; adopted from Vanderbild data set.

\textbf{parch} an ordered factor specifying the number of
parents/children aboard; adopted from Vanderbild data set.

\textbf{survived} a factor with two levels (no and yes) specifying
whether the person has survived the sinking.

\hypertarget{anuxe1lisis-de-los-datos}{%
\subsection{Análisis de los datos}\label{anuxe1lisis-de-los-datos}}

\begin{itemize}
\tightlist
\item
  Ahora procedemos a sacar algunas estadísticas básicas y después
  analizaremos los atributos con valores vacíos.
\end{itemize}

\begin{Shaded}
\begin{Highlighting}[]
\FunctionTok{summary}\NormalTok{(totalData)}
\end{Highlighting}
\end{Shaded}

\begin{verbatim}
##      name              gender               age             class          
##  Length:2207        Length:2207        Min.   : 0.1667   Length:2207       
##  Class :character   Class :character   1st Qu.:22.0000   Class :character  
##  Mode  :character   Mode  :character   Median :29.0000   Mode  :character  
##                                        Mean   :30.4367                     
##                                        3rd Qu.:38.0000                     
##                                        Max.   :74.0000                     
##                                        NA's   :2                           
##    embarked           country             ticketno            fare        
##  Length:2207        Length:2207        Min.   :      2   Min.   :  3.030  
##  Class :character   Class :character   1st Qu.:  14262   1st Qu.:  7.181  
##  Mode  :character   Mode  :character   Median : 111427   Median : 14.090  
##                                        Mean   : 284216   Mean   : 33.405  
##                                        3rd Qu.: 347077   3rd Qu.: 31.061  
##                                        Max.   :3101317   Max.   :512.061  
##                                        NA's   :891       NA's   :916      
##      sibsp            parch          survived        
##  Min.   :0.0000   Min.   :0.0000   Length:2207       
##  1st Qu.:0.0000   1st Qu.:0.0000   Class :character  
##  Median :0.0000   Median :0.0000   Mode  :character  
##  Mean   :0.4996   Mean   :0.3856                     
##  3rd Qu.:1.0000   3rd Qu.:0.0000                     
##  Max.   :8.0000   Max.   :9.0000                     
##  NA's   :900      NA's   :900
\end{verbatim}

\begin{itemize}
\tightlist
\item
  Por ejemplo estadísticas de valores vacíos.
\end{itemize}

\begin{Shaded}
\begin{Highlighting}[]
\FunctionTok{colSums}\NormalTok{(}\FunctionTok{is.na}\NormalTok{(totalData))}
\end{Highlighting}
\end{Shaded}

\begin{verbatim}
##     name   gender      age    class embarked  country ticketno     fare 
##        0        0        2        0        0       81      891      916 
##    sibsp    parch survived 
##      900      900        0
\end{verbatim}

\begin{Shaded}
\begin{Highlighting}[]
\FunctionTok{colSums}\NormalTok{(totalData}\SpecialCharTok{==}\StringTok{""}\NormalTok{)}
\end{Highlighting}
\end{Shaded}

\begin{verbatim}
##     name   gender      age    class embarked  country ticketno     fare 
##        0        0       NA        0        0       NA       NA       NA 
##    sibsp    parch survived 
##       NA       NA        0
\end{verbatim}

\begin{itemize}
\tightlist
\item
  Para estos casos (81) de país, asignamos valor ``Desconocido'' para
  los valores vacíos de la variable ``country''.
\end{itemize}

\begin{Shaded}
\begin{Highlighting}[]
\NormalTok{totalData}\SpecialCharTok{$}\NormalTok{country[}\FunctionTok{is.na}\NormalTok{(totalData}\SpecialCharTok{$}\NormalTok{country)] }\OtherTok{\textless{}{-}} \StringTok{"Desconocido"}
\end{Highlighting}
\end{Shaded}

\begin{itemize}
\tightlist
\item
  Para el caso de edad (2), asignamos la media para valores vacíos de la
  variable ``age''.
\end{itemize}

\begin{Shaded}
\begin{Highlighting}[]
\NormalTok{totalData}\SpecialCharTok{$}\NormalTok{age[}\FunctionTok{is.na}\NormalTok{(totalData}\SpecialCharTok{$}\NormalTok{age)] }\OtherTok{\textless{}{-}} \FunctionTok{mean}\NormalTok{(totalData}\SpecialCharTok{$}\NormalTok{age,}\AttributeTok{na.rm=}\NormalTok{T)}
\end{Highlighting}
\end{Shaded}

De la información mostrada destacamos que el pasajero más joven tenía 6
meses y el más grande 74 años. La media de edad la tenían en 30 años.
También podemos ver 891 sin billete. Revisaremos si se corresponde a la
tripulación. También podemos observar el que se pagó por el billete. En
este caso se entienden las discrepancias en la fiabilidad de este dato.
Parece que los pasajeros que embarcaron a Southampton hacían transbordo
de un barco que tenía la tripulación en huelga y por eso no tuvieron que
pagar lo que explicaría la diferencia. Recordemos que la tripulación no
pagaba. Sibsp y parch también muestran datos interesantes el viajero con
quien más familiar viajaba eran 8 hermanos o mujer y 9 hijos o
paro/madre.

Si observamos los NA (valores nulos) vemos que los datos están bastante
bien. Decidimos sustituir el valor NA de country por Desconocido por una
mayor legibilidad. También proponemos sustituir los NA de age por la
media a pesar de que realmente no hace falta.

Es curios como los valores NA de sibsp y parch nos permite deducir que
viajaban muchas familias. De hecho a simple vista, restante la
tripulación la gente que viajaba sola era mínima. Este dato la podríamos
contrastar también. Sería interesante relacionar la mortalidad del
accidente con el tamaño de las familias que viajaban.

\hypertarget{pruebas-estaduxedsticas}{%
\subsection{Pruebas estadísticas}\label{pruebas-estaduxedsticas}}

\begin{itemize}
\tightlist
\item
  Ahora añadiremos un campo nuevo a los datos. Este campos contendrá el
  valor de la edad discretitzada con un método simple de intervalos de
  igual amplitud.
\end{itemize}

\begin{Shaded}
\begin{Highlighting}[]
\FunctionTok{summary}\NormalTok{(totalData[,}\StringTok{"age"}\NormalTok{])}
\end{Highlighting}
\end{Shaded}

\begin{verbatim}
##    Min. 1st Qu.  Median    Mean 3rd Qu.    Max. 
##  0.1667 22.0000 29.0000 30.4367 38.0000 74.0000
\end{verbatim}

\begin{itemize}
\tightlist
\item
  Procedemos a discretizar con intervalos.
\end{itemize}

\begin{Shaded}
\begin{Highlighting}[]
\NormalTok{totalData[}\StringTok{"segmento\_edad"}\NormalTok{] }\OtherTok{\textless{}{-}} \FunctionTok{cut}\NormalTok{(totalData}\SpecialCharTok{$}\NormalTok{age, }\AttributeTok{breaks =} \FunctionTok{c}\NormalTok{(}\DecValTok{0}\NormalTok{,}\DecValTok{10}\NormalTok{,}\DecValTok{20}\NormalTok{,}\DecValTok{30}\NormalTok{,}\DecValTok{40}\NormalTok{,}\DecValTok{50}\NormalTok{,}\DecValTok{60}\NormalTok{,}\DecValTok{70}\NormalTok{,}\DecValTok{100}\NormalTok{), }\AttributeTok{labels =} \FunctionTok{c}\NormalTok{(}\StringTok{"0{-}9"}\NormalTok{, }\StringTok{"10{-}19"}\NormalTok{, }\StringTok{"20{-}29"}\NormalTok{, }\StringTok{"30{-}39"}\NormalTok{,}\StringTok{"40{-}49"}\NormalTok{,}\StringTok{"50{-}59"}\NormalTok{,}\StringTok{"60{-}69"}\NormalTok{,}\StringTok{"70{-}79"}\NormalTok{))}
\end{Highlighting}
\end{Shaded}

\begin{itemize}
\tightlist
\item
  Y Observamos los datos discretizados.
\end{itemize}

\begin{Shaded}
\begin{Highlighting}[]
\FunctionTok{head}\NormalTok{(totalData)}
\end{Highlighting}
\end{Shaded}

\begin{verbatim}
##                             name gender age class embarked       country
## 1            Abbing, Mr. Anthony   male  42   3rd        S United States
## 2      Abbott, Mr. Eugene Joseph   male  13   3rd        S United States
## 3    Abbott, Mr. Rossmore Edward   male  16   3rd        S United States
## 4 Abbott, Mrs. Rhoda Mary 'Rosa' female  39   3rd        S       England
## 5    Abelseth, Miss. Karen Marie female  16   3rd        S        Norway
## 6 Abelseth, Mr. Olaus Jørgensen   male  25   3rd        S United States
##   ticketno  fare sibsp parch survived segmento_edad
## 1     5547  7.11     0     0       no         40-49
## 2     2673 20.05     0     2       no         10-19
## 3     2673 20.05     1     1       no         10-19
## 4     2673 20.05     1     1      yes         30-39
## 5   348125  7.13     0     0      yes         10-19
## 6   348122  7.13     0     0      yes         20-29
\end{verbatim}

\begin{itemize}
\tightlist
\item
  Ahora podemos ver como se agrupaban por grupos de edad.
\end{itemize}

\begin{Shaded}
\begin{Highlighting}[]
\FunctionTok{plot}\NormalTok{(totalData}\SpecialCharTok{$}\NormalTok{segmento\_edad,}\AttributeTok{main=}\StringTok{"Número de pasajeros por grupos de edad"}\NormalTok{,}\AttributeTok{xlab=}\StringTok{"Edad"}\NormalTok{, }\AttributeTok{ylab=}\StringTok{"Cantidad"}\NormalTok{,}\AttributeTok{col =} \StringTok{"ivory"}\NormalTok{)}
\end{Highlighting}
\end{Shaded}

\includegraphics{LimpiezaAnalisis_Titanic_files/figure-latex/unnamed-chunk-12-1.pdf}

\begin{itemize}
\tightlist
\item
  Procedemos a repetir los pasos anteriores pero solo sobre el
  subconjunto de tripulación filtrado antes ``engineering crew''.
\end{itemize}

\begin{Shaded}
\begin{Highlighting}[]
\NormalTok{totalData\_crew[}\StringTok{"segmento\_edad"}\NormalTok{] }\OtherTok{\textless{}{-}} \FunctionTok{cut}\NormalTok{(totalData\_crew}\SpecialCharTok{$}\NormalTok{age, }\AttributeTok{breaks =} \FunctionTok{c}\NormalTok{(}\DecValTok{0}\NormalTok{,}\DecValTok{10}\NormalTok{,}\DecValTok{20}\NormalTok{,}\DecValTok{30}\NormalTok{,}\DecValTok{40}\NormalTok{,}\DecValTok{50}\NormalTok{,}\DecValTok{60}\NormalTok{,}\DecValTok{70}\NormalTok{,}\DecValTok{100}\NormalTok{), }\AttributeTok{labels =} \FunctionTok{c}\NormalTok{(}\StringTok{"0{-}9"}\NormalTok{, }\StringTok{"10{-}19"}\NormalTok{, }\StringTok{"20{-}29"}\NormalTok{, }\StringTok{"30{-}39"}\NormalTok{,}\StringTok{"40{-}49"}\NormalTok{,}\StringTok{"50{-}59"}\NormalTok{,}\StringTok{"60{-}69"}\NormalTok{,}\StringTok{"70{-}79"}\NormalTok{))}
\FunctionTok{plot}\NormalTok{(totalData\_crew}\SpecialCharTok{$}\NormalTok{segmento\_edad,}\AttributeTok{main=}\StringTok{"Número de tripulantes por grupos de edad"}\NormalTok{,}\AttributeTok{xlab=}\StringTok{"Edad"}\NormalTok{, }\AttributeTok{ylab=}\StringTok{"Cantidad"}\NormalTok{,}\AttributeTok{col =} \StringTok{"ivory"}\NormalTok{)}
\end{Highlighting}
\end{Shaded}

\includegraphics{LimpiezaAnalisis_Titanic_files/figure-latex/unnamed-chunk-13-1.pdf}
De la discretización de la edad observamos que realmente la gente que
viajaba era muy joven. El segmento más grande era de 20 a 29 años.
También podemos observar la juventud de la tripulación del crucero.

\begin{itemize}
\tightlist
\item
  Como alternativa a la discretización realizada discretizaremos ahora
  edad con kmeans.
\end{itemize}

\begin{Shaded}
\begin{Highlighting}[]
\CommentTok{\# https://cran.r{-}project.org/web/packages/arules/index.html}
\ControlFlowTok{if}\NormalTok{ (}\SpecialCharTok{!}\FunctionTok{require}\NormalTok{(}\StringTok{\textquotesingle{}arules\textquotesingle{}}\NormalTok{)) }\FunctionTok{install.packages}\NormalTok{(}\StringTok{\textquotesingle{}arules\textquotesingle{}}\NormalTok{); }\FunctionTok{library}\NormalTok{(}\StringTok{\textquotesingle{}arules\textquotesingle{}}\NormalTok{)}
\end{Highlighting}
\end{Shaded}

\begin{verbatim}
## Loading required package: arules
\end{verbatim}

\begin{verbatim}
## Loading required package: Matrix
\end{verbatim}

\begin{verbatim}
## 
## Attaching package: 'arules'
\end{verbatim}

\begin{verbatim}
## The following object is masked from 'package:dplyr':
## 
##     recode
\end{verbatim}

\begin{verbatim}
## The following objects are masked from 'package:base':
## 
##     abbreviate, write
\end{verbatim}

\begin{Shaded}
\begin{Highlighting}[]
\FunctionTok{set.seed}\NormalTok{(}\DecValTok{2}\NormalTok{)}
\FunctionTok{table}\NormalTok{(}\FunctionTok{discretize}\NormalTok{(totalData}\SpecialCharTok{$}\NormalTok{age, }\StringTok{"cluster"}\NormalTok{ ))}
\end{Highlighting}
\end{Shaded}

\begin{verbatim}
## 
## [0.167,25.4)    [25.4,40)      [40,74] 
##          826          916          465
\end{verbatim}

\begin{Shaded}
\begin{Highlighting}[]
\FunctionTok{hist}\NormalTok{(totalData}\SpecialCharTok{$}\NormalTok{age, }\AttributeTok{main=}\StringTok{"Número de pasajeros por grupos de edad con kmeans"}\NormalTok{,}\AttributeTok{xlab=}\StringTok{"Edad"}\NormalTok{, }\AttributeTok{ylab=}\StringTok{"Cantidad"}\NormalTok{,}\AttributeTok{col =} \StringTok{"ivory"}\NormalTok{)}
\FunctionTok{abline}\NormalTok{(}\AttributeTok{v=}\FunctionTok{discretize}\NormalTok{(totalData}\SpecialCharTok{$}\NormalTok{age, }\AttributeTok{method=}\StringTok{"cluster"}\NormalTok{, }\AttributeTok{onlycuts=}\ConstantTok{TRUE}\NormalTok{),}\AttributeTok{col=}\StringTok{"red"}\NormalTok{)}
\end{Highlighting}
\end{Shaded}

\includegraphics{LimpiezaAnalisis_Titanic_files/figure-latex/unnamed-chunk-14-1.pdf}

\begin{itemize}
\tightlist
\item
  Podemos observar que sin pasar ningún argumento y que el algoritmo
  escoja el conjunto de particiones se muestran tres clústeres que
  agrupan las edades en las franjas mencionadas. Podemos asignar el
  propio clúster como una variable más al dataset para trabajar después.
\end{itemize}

\begin{Shaded}
\begin{Highlighting}[]
\NormalTok{totalData}\SpecialCharTok{$}\NormalTok{edad\_KM }\OtherTok{\textless{}{-}}\NormalTok{ (}\FunctionTok{discretize}\NormalTok{(totalData}\SpecialCharTok{$}\NormalTok{age, }\StringTok{"cluster"}\NormalTok{ ))}
\FunctionTok{head}\NormalTok{(totalData)}
\end{Highlighting}
\end{Shaded}

\begin{verbatim}
##                             name gender age class embarked       country
## 1            Abbing, Mr. Anthony   male  42   3rd        S United States
## 2      Abbott, Mr. Eugene Joseph   male  13   3rd        S United States
## 3    Abbott, Mr. Rossmore Edward   male  16   3rd        S United States
## 4 Abbott, Mrs. Rhoda Mary 'Rosa' female  39   3rd        S       England
## 5    Abelseth, Miss. Karen Marie female  16   3rd        S        Norway
## 6 Abelseth, Mr. Olaus Jørgensen   male  25   3rd        S United States
##   ticketno  fare sibsp parch survived segmento_edad      edad_KM
## 1     5547  7.11     0     0       no         40-49    [38.7,74]
## 2     2673 20.05     0     2       no         10-19 [0.167,23.9)
## 3     2673 20.05     1     1       no         10-19 [0.167,23.9)
## 4     2673 20.05     1     1      yes         30-39    [38.7,74]
## 5   348125  7.13     0     0      yes         10-19 [0.167,23.9)
## 6   348122  7.13     0     0      yes         20-29  [23.9,38.7)
\end{verbatim}

\begin{itemize}
\tightlist
\item
  Ahora normalizaremos la edad de los pasajeros por el máximo, añadiendo
  un nuevo valor a los datos que contendrá el valor.
\end{itemize}

\begin{Shaded}
\begin{Highlighting}[]
\NormalTok{totalData}\SpecialCharTok{$}\NormalTok{age\_NM }\OtherTok{\textless{}{-}}\NormalTok{ (totalData}\SpecialCharTok{$}\NormalTok{age}\SpecialCharTok{/}\FunctionTok{max}\NormalTok{(totalData[,}\StringTok{"age"}\NormalTok{]))}
\FunctionTok{head}\NormalTok{(totalData}\SpecialCharTok{$}\NormalTok{age\_NM)}
\end{Highlighting}
\end{Shaded}

\begin{verbatim}
## [1] 0.5675676 0.1756757 0.2162162 0.5270270 0.2162162 0.3378378
\end{verbatim}

\begin{itemize}
\tightlist
\item
  Supongamos que queremos normalizar por la diferencia para ubicar entre
  0 y 1 la variable edad del pasajero dado que el algoritmo de minería
  que utilizaremos así lo requiere. observamos la distribución de la
  variable original y las tres generadas
\end{itemize}

\begin{Shaded}
\begin{Highlighting}[]
\NormalTok{totalData}\SpecialCharTok{$}\NormalTok{age\_ND }\OtherTok{=}\NormalTok{ (totalData}\SpecialCharTok{$}\NormalTok{age}\SpecialCharTok{{-}}\FunctionTok{min}\NormalTok{(totalData}\SpecialCharTok{$}\NormalTok{age))}\SpecialCharTok{/}\NormalTok{(}\FunctionTok{max}\NormalTok{(totalData}\SpecialCharTok{$}\NormalTok{age)}\SpecialCharTok{{-}}\FunctionTok{min}\NormalTok{(totalData}\SpecialCharTok{$}\NormalTok{age))}

\FunctionTok{max}\NormalTok{(totalData}\SpecialCharTok{$}\NormalTok{age)}
\end{Highlighting}
\end{Shaded}

\begin{verbatim}
## [1] 74
\end{verbatim}

\begin{Shaded}
\begin{Highlighting}[]
\FunctionTok{min}\NormalTok{(totalData}\SpecialCharTok{$}\NormalTok{age)}
\end{Highlighting}
\end{Shaded}

\begin{verbatim}
## [1] 0.1666667
\end{verbatim}

\begin{Shaded}
\begin{Highlighting}[]
\FunctionTok{hist}\NormalTok{(totalData}\SpecialCharTok{$}\NormalTok{age,}\AttributeTok{xlab=}\StringTok{"Edad"}\NormalTok{, }\AttributeTok{col=}\StringTok{"ivory"}\NormalTok{,}\AttributeTok{ylab=}\StringTok{"Cantidad"}\NormalTok{, }\AttributeTok{main=}\StringTok{"Número de pasajeros por grupos de edad"}\NormalTok{)}
\end{Highlighting}
\end{Shaded}

\includegraphics{LimpiezaAnalisis_Titanic_files/figure-latex/unnamed-chunk-17-1.pdf}

\begin{Shaded}
\begin{Highlighting}[]
\FunctionTok{hist}\NormalTok{(totalData}\SpecialCharTok{$}\NormalTok{age\_NM,}\AttributeTok{xlab=}\StringTok{"Edad normalizada por el máximo"}\NormalTok{, }\AttributeTok{ylab=}\StringTok{"Cantidad"}\NormalTok{,}\AttributeTok{col=}\StringTok{"ivory"}\NormalTok{, }\AttributeTok{main=}\StringTok{"Número de pasajeros por grupos de edad"}\NormalTok{)}
\end{Highlighting}
\end{Shaded}

\includegraphics{LimpiezaAnalisis_Titanic_files/figure-latex/unnamed-chunk-17-2.pdf}

\begin{Shaded}
\begin{Highlighting}[]
\FunctionTok{hist}\NormalTok{(totalData}\SpecialCharTok{$}\NormalTok{age\_ND,}\AttributeTok{xlab=}\StringTok{"Edad normalizada por la diferencia"}\NormalTok{,}\AttributeTok{ylab=}\StringTok{"Cantidad"}\NormalTok{, }\AttributeTok{col=}\StringTok{"ivory"}\NormalTok{, }\AttributeTok{main=}\StringTok{"Número de pasajeros por grupos de edad"}\NormalTok{)}
\end{Highlighting}
\end{Shaded}

\includegraphics{LimpiezaAnalisis_Titanic_files/figure-latex/unnamed-chunk-17-3.pdf}

\hypertarget{procesos-de-anuxe1lisis-visuales-del-juego-de-datos}{%
\subsection{Procesos de análisis visuales del juego de
datos}\label{procesos-de-anuxe1lisis-visuales-del-juego-de-datos}}

\begin{itemize}
\tightlist
\item
  Nos proponemos analizar las relaciones entre las diferentes variables
  del juego de datos para ver si se relacionan y como. Visualizamos la
  relación entre las variables ``gender'' y ``survived'':
\end{itemize}

\begin{Shaded}
\begin{Highlighting}[]
\FunctionTok{ggplot}\NormalTok{(}\AttributeTok{data=}\NormalTok{totalData[}\DecValTok{1}\SpecialCharTok{:}\NormalTok{filas,],}\FunctionTok{aes}\NormalTok{(}\AttributeTok{x=}\NormalTok{gender,}\AttributeTok{fill=}\NormalTok{survived))}\SpecialCharTok{+}\FunctionTok{geom\_bar}\NormalTok{()}\SpecialCharTok{+}\FunctionTok{ggtitle}\NormalTok{(}\StringTok{"Relación entre las variables gender y survived"}\NormalTok{)}
\end{Highlighting}
\end{Shaded}

\includegraphics{LimpiezaAnalisis_Titanic_files/figure-latex/unnamed-chunk-18-1.pdf}
* Otro punto de vista. Survived como función de Embarked:

\begin{Shaded}
\begin{Highlighting}[]
\FunctionTok{ggplot}\NormalTok{(}\AttributeTok{data=}\NormalTok{totalData[}\DecValTok{1}\SpecialCharTok{:}\NormalTok{filas,],}\FunctionTok{aes}\NormalTok{(}\AttributeTok{x=}\NormalTok{embarked,}\AttributeTok{fill=}\NormalTok{survived))}\SpecialCharTok{+}\FunctionTok{geom\_bar}\NormalTok{(}\AttributeTok{position=}\StringTok{"fill"}\NormalTok{)}\SpecialCharTok{+}\FunctionTok{ylab}\NormalTok{(}\StringTok{"Frequència"}\NormalTok{)}\SpecialCharTok{+}\FunctionTok{ggtitle}\NormalTok{(}\StringTok{"Survived como función de Embarked"}\NormalTok{)}
\end{Highlighting}
\end{Shaded}

\includegraphics{LimpiezaAnalisis_Titanic_files/figure-latex/unnamed-chunk-19-1.pdf}

\begin{itemize}
\item
  En la primera gráfica podemos observar fácilmente la cantidad de
  mujeres que viajaban respecto hombres y observar los que no
  sobrevivieron. Numéricamente el número de hombres y mujeres
  supervivientes es similar.
\item
  En la segunda gráfica de forma porcentual observamos los puertos de
  embarque y los porcentajes de supervivencia en función del puerto. Se
  podría trabajar el puerto C (Cherburgo) para ver de explicar la
  diferencia en los datos. Quizás porcentualmente embarcaron más mujeres
  o niños\ldots{} ¿O gente de primera clase?
\end{itemize}

*Obtenemos ahora una matriz de porcentajes de frecuencia. Vemos, por
ejemplo que la probabilidad de sobrevivir si se embarcó en ``C'' es de
un 56.45\%

\begin{Shaded}
\begin{Highlighting}[]
\NormalTok{t}\OtherTok{\textless{}{-}}\FunctionTok{table}\NormalTok{(totalData[}\DecValTok{1}\SpecialCharTok{:}\NormalTok{filas,]}\SpecialCharTok{$}\NormalTok{embarked,totalData[}\DecValTok{1}\SpecialCharTok{:}\NormalTok{filas,]}\SpecialCharTok{$}\NormalTok{survived)}
\ControlFlowTok{for}\NormalTok{ (i }\ControlFlowTok{in} \DecValTok{1}\SpecialCharTok{:}\FunctionTok{dim}\NormalTok{(t)[}\DecValTok{1}\NormalTok{])\{}
\NormalTok{    t[i,]}\OtherTok{\textless{}{-}}\NormalTok{t[i,]}\SpecialCharTok{/}\FunctionTok{sum}\NormalTok{(t[i,])}\SpecialCharTok{*}\DecValTok{100}
\NormalTok{\}}
\NormalTok{t}
\end{Highlighting}
\end{Shaded}

\begin{verbatim}
##    
##           no      yes
##   B 78.17259 21.82741
##   C 43.54244 56.45756
##   Q 64.22764 35.77236
##   S 70.85396 29.14604
\end{verbatim}

\begin{itemize}
\tightlist
\item
  Veamos ahora como en un mismo gráfico de frecuencias podemos trabajar
  con 3 variables: Embarked, Survived y class. Mostramos el gráfico de
  embarcados por class:
\end{itemize}

\begin{Shaded}
\begin{Highlighting}[]
\FunctionTok{ggplot}\NormalTok{(}\AttributeTok{data =}\NormalTok{ totalData[}\DecValTok{1}\SpecialCharTok{:}\NormalTok{filas,],}\FunctionTok{aes}\NormalTok{(}\AttributeTok{x=}\NormalTok{embarked,}\AttributeTok{fill=}\NormalTok{survived))}\SpecialCharTok{+}\FunctionTok{geom\_bar}\NormalTok{(}\AttributeTok{position=}\StringTok{"fill"}\NormalTok{)}\SpecialCharTok{+}\FunctionTok{facet\_wrap}\NormalTok{(}\SpecialCharTok{\textasciitilde{}}\NormalTok{class)}\SpecialCharTok{+}\FunctionTok{ggtitle}\NormalTok{(}\StringTok{"Pasajeros por clase, puerto de origen y relación con survived"}\NormalTok{)}
\end{Highlighting}
\end{Shaded}

\includegraphics{LimpiezaAnalisis_Titanic_files/figure-latex/unnamed-chunk-21-1.pdf}

\begin{itemize}
\item
  Aquí ya podemos extraer mucha información. Como propuesta de mejora se
  podría hacer un gráfico similar trabajando solo la clase. Habría que
  unificar toda la tripulación a una única categoría.
\item
  Comparamos ahora dos gráficos de frecuencias: Survived-SibSp y
  Survived-Parch
\end{itemize}

\begin{Shaded}
\begin{Highlighting}[]
\FunctionTok{ggplot}\NormalTok{(}\AttributeTok{data =}\NormalTok{ totalData[}\DecValTok{1}\SpecialCharTok{:}\NormalTok{filas,],}\FunctionTok{aes}\NormalTok{(}\AttributeTok{x=}\NormalTok{sibsp,}\AttributeTok{fill=}\NormalTok{survived))}\SpecialCharTok{+}\FunctionTok{geom\_bar}\NormalTok{()}\SpecialCharTok{+}\FunctionTok{ggtitle}\NormalTok{(}\StringTok{"Sobrevivir en función de tener a bordo cónyuges y/o hermanos"}\NormalTok{)}
\end{Highlighting}
\end{Shaded}

\includegraphics{LimpiezaAnalisis_Titanic_files/figure-latex/unnamed-chunk-22-1.pdf}

\begin{Shaded}
\begin{Highlighting}[]
\FunctionTok{ggplot}\NormalTok{(}\AttributeTok{data =}\NormalTok{ totalData[}\DecValTok{1}\SpecialCharTok{:}\NormalTok{filas,],}\FunctionTok{aes}\NormalTok{(}\AttributeTok{x=}\NormalTok{parch,}\AttributeTok{fill=}\NormalTok{survived))}\SpecialCharTok{+}\FunctionTok{geom\_bar}\NormalTok{()}\SpecialCharTok{+}\FunctionTok{ggtitle}\NormalTok{(}\StringTok{"Sobrevivir en función de tener a bordo padres y/o hijos"}\NormalTok{)}
\end{Highlighting}
\end{Shaded}

\includegraphics{LimpiezaAnalisis_Titanic_files/figure-latex/unnamed-chunk-22-2.pdf}

\begin{itemize}
\item
  Vemos como la forma de estos dos gráficos es similar. Este hecho nos
  puede indicar presencia de correlaciones altas. Hecho previsible en
  función de la descripción de las variables.
\item
  Veamos un ejemplo de construcción de una variable nueva: Tamaño de
  familia.
\end{itemize}

\begin{Shaded}
\begin{Highlighting}[]
\NormalTok{totalData}\SpecialCharTok{$}\NormalTok{FamilySize }\OtherTok{\textless{}{-}}\NormalTok{ totalData}\SpecialCharTok{$}\NormalTok{sibsp }\SpecialCharTok{+}\NormalTok{ totalData}\SpecialCharTok{$}\NormalTok{parch }\SpecialCharTok{+}\DecValTok{1}\NormalTok{;}
\NormalTok{totalData1}\OtherTok{\textless{}{-}}\NormalTok{totalData[}\DecValTok{1}\SpecialCharTok{:}\NormalTok{filas,]}
\FunctionTok{ggplot}\NormalTok{(}\AttributeTok{data =}\NormalTok{ totalData1[}\SpecialCharTok{!}\FunctionTok{is.na}\NormalTok{(totalData[}\DecValTok{1}\SpecialCharTok{:}\NormalTok{filas,]}\SpecialCharTok{$}\NormalTok{FamilySize),],}\FunctionTok{aes}\NormalTok{(}\AttributeTok{x=}\NormalTok{FamilySize,}\AttributeTok{fill=}\NormalTok{survived))}\SpecialCharTok{+}\FunctionTok{geom\_histogram}\NormalTok{(}\AttributeTok{binwidth =}\DecValTok{1}\NormalTok{,}\AttributeTok{position=}\StringTok{"fill"}\NormalTok{)}\SpecialCharTok{+}\FunctionTok{ylab}\NormalTok{(}\StringTok{"Frecuencia"}\NormalTok{)}\SpecialCharTok{+}\FunctionTok{ggtitle}\NormalTok{(}\StringTok{"Sobrevivir en función del número de familiares a bordo"}\NormalTok{)}
\end{Highlighting}
\end{Shaded}

\includegraphics{LimpiezaAnalisis_Titanic_files/figure-latex/unnamed-chunk-23-1.pdf}
* Se confirma el hecho de que los pasajeros viajaban mayoritariamente en
familia. No podemos afirmar que el tamaño de la familia tuviera nada que
ver con la posibilidad de sobrevivir pues nos tememos que
estadísticamente el hecho de haber más familias de alrededor de cuatro
miembros debería de ser habitual. Es un punto de partida para investigar
más.

\begin{itemize}
\tightlist
\item
  Veamos ahora dos gráficos que nos comparan los atributos Age y
  Survived. Observamos como el parámetro position=``fill'' nos da la
  proporción acumulada de un atributo dentro de otro.
\end{itemize}

\begin{Shaded}
\begin{Highlighting}[]
\FunctionTok{ggplot}\NormalTok{(}\AttributeTok{data =}\NormalTok{ totalData1[}\SpecialCharTok{!}\NormalTok{(}\FunctionTok{is.na}\NormalTok{(totalData[}\DecValTok{1}\SpecialCharTok{:}\NormalTok{filas,]}\SpecialCharTok{$}\NormalTok{age)),],}\FunctionTok{aes}\NormalTok{(}\AttributeTok{x=}\NormalTok{age,}\AttributeTok{fill=}\NormalTok{survived))}\SpecialCharTok{+}\FunctionTok{geom\_histogram}\NormalTok{(}\AttributeTok{binwidth =}\DecValTok{3}\NormalTok{)}\SpecialCharTok{+}\FunctionTok{ggtitle}\NormalTok{(}\StringTok{"Sobrevivir en función de edad"}\NormalTok{)}
\end{Highlighting}
\end{Shaded}

\includegraphics{LimpiezaAnalisis_Titanic_files/figure-latex/unnamed-chunk-24-1.pdf}

\begin{Shaded}
\begin{Highlighting}[]
\FunctionTok{ggplot}\NormalTok{(}\AttributeTok{data =}\NormalTok{ totalData1[}\SpecialCharTok{!}\FunctionTok{is.na}\NormalTok{(totalData[}\DecValTok{1}\SpecialCharTok{:}\NormalTok{filas,]}\SpecialCharTok{$}\NormalTok{age),],}\FunctionTok{aes}\NormalTok{(}\AttributeTok{x=}\NormalTok{age,}\AttributeTok{fill=}\NormalTok{survived))}\SpecialCharTok{+}\FunctionTok{geom\_histogram}\NormalTok{(}\AttributeTok{binwidth =} \DecValTok{3}\NormalTok{,}\AttributeTok{position=}\StringTok{"fill"}\NormalTok{)}\SpecialCharTok{+}\FunctionTok{ylab}\NormalTok{(}\StringTok{"Frecuencia"}\NormalTok{)}\SpecialCharTok{+}\FunctionTok{ggtitle}\NormalTok{(}\StringTok{"Sobrevivir en función de edad"}\NormalTok{)}
\end{Highlighting}
\end{Shaded}

\includegraphics{LimpiezaAnalisis_Titanic_files/figure-latex/unnamed-chunk-24-2.pdf}

\begin{itemize}
\item
  Observamos como el parámetro position=``hijo'' nos da la proporción
  acumulada de un atributo dentro de otro. Parece que los niños tuvieron
  más posibilidad de salvarse.
\item
  Vamos a probar si hay una correlación entre la edad del pasajero y el
  que pagó por el viaje.
\end{itemize}

\begin{Shaded}
\begin{Highlighting}[]
\CommentTok{\# https://cran.r{-}project.org/web/packages/tidyverse/index.html}
\ControlFlowTok{if}\NormalTok{ (}\SpecialCharTok{!}\FunctionTok{require}\NormalTok{(}\StringTok{\textquotesingle{}tidyverse\textquotesingle{}}\NormalTok{)) }\FunctionTok{install.packages}\NormalTok{(}\StringTok{\textquotesingle{}tidyverse\textquotesingle{}}\NormalTok{); }\FunctionTok{library}\NormalTok{(}\StringTok{\textquotesingle{}tidyverse\textquotesingle{}}\NormalTok{)}
\end{Highlighting}
\end{Shaded}

\begin{verbatim}
## Loading required package: tidyverse
\end{verbatim}

\begin{verbatim}
## -- Attaching packages --------------------------------------- tidyverse 1.3.1 --
\end{verbatim}

\begin{verbatim}
## v tibble  3.1.5     v purrr   0.3.4
## v tidyr   1.1.4     v stringr 1.4.0
## v readr   2.0.2     v forcats 0.5.1
\end{verbatim}

\begin{verbatim}
## -- Conflicts ------------------------------------------ tidyverse_conflicts() --
## x tidyr::expand()  masks Matrix::expand()
## x dplyr::filter()  masks stats::filter()
## x dplyr::lag()     masks stats::lag()
## x tidyr::pack()    masks Matrix::pack()
## x arules::recode() masks dplyr::recode()
## x tidyr::unpack()  masks Matrix::unpack()
\end{verbatim}

\begin{Shaded}
\begin{Highlighting}[]
\FunctionTok{cor.test}\NormalTok{(}\AttributeTok{x =}\NormalTok{ totalData}\SpecialCharTok{$}\NormalTok{age, }\AttributeTok{y =}\NormalTok{ totalData}\SpecialCharTok{$}\NormalTok{fare, }\AttributeTok{method =} \StringTok{"pearson"}\NormalTok{)}
\end{Highlighting}
\end{Shaded}

\begin{verbatim}
## 
##  Pearson's product-moment correlation
## 
## data:  totalData$age and totalData$fare
## t = 6.7199, df = 1289, p-value = 2.722e-11
## alternative hypothesis: true correlation is not equal to 0
## 95 percent confidence interval:
##  0.1307297 0.2361631
## sample estimates:
##       cor 
## 0.1839756
\end{verbatim}

\begin{Shaded}
\begin{Highlighting}[]
\FunctionTok{ggplot}\NormalTok{(}\AttributeTok{data =}\NormalTok{ totalData, }\FunctionTok{aes}\NormalTok{(}\AttributeTok{x =}\NormalTok{ age, }\AttributeTok{y =} \FunctionTok{log}\NormalTok{(fare))) }\SpecialCharTok{+} \FunctionTok{geom\_point}\NormalTok{(}\AttributeTok{color =} \StringTok{"gray30"}\NormalTok{) }\SpecialCharTok{+} \FunctionTok{geom\_smooth}\NormalTok{(}\AttributeTok{color =} \StringTok{"firebrick"}\NormalTok{) }\SpecialCharTok{+} \FunctionTok{theme\_bw}\NormalTok{() }\SpecialCharTok{+}\FunctionTok{ggtitle}\NormalTok{(}\StringTok{"Correlación entre precio billete y edad"}\NormalTok{)}
\end{Highlighting}
\end{Shaded}

\begin{verbatim}
## `geom_smooth()` using method = 'gam' and formula 'y ~ s(x, bs = "cs")'
\end{verbatim}

\begin{verbatim}
## Warning: Removed 916 rows containing non-finite values (stat_smooth).
\end{verbatim}

\begin{verbatim}
## Warning: Removed 916 rows containing missing values (geom_point).
\end{verbatim}

\includegraphics{LimpiezaAnalisis_Titanic_files/figure-latex/unnamed-chunk-25-1.pdf}

\begin{itemize}
\tightlist
\item
  Cómo podemos observar no parece haber correlación lineal entre la edad
  del pasajero y el precio del billete. El diagrama de dispersión
  tampoco apunta a ningún tipo de relación no lineal evidente.
\end{itemize}

\hypertarget{conclusiones}{%
\subsection{Conclusiones}\label{conclusiones}}

Los datos tienen una calidad correcta y están mayoritariamente bien
informados. Disponen de una variable de clase ``survived'' que los hace
aptos para un clasificador. A parte de la mayor supervivencia de mujeres
y niños y de pasajeros de primera clase podemos observar la juventud de
los pasajeros y la tripulación. Se observa también una gran cantidad de
personas que viajaban en familia.

\begin{center}\rule{0.5\linewidth}{0.5pt}\end{center}

\hypertarget{recursos}{%
\section{Recursos}\label{recursos}}

\begin{center}\rule{0.5\linewidth}{0.5pt}\end{center}

Los siguientes recursos son de utilidad para la realización de la
práctica: * Calvo M., Subirats L., Pérez D. (2019). Introducción a la
limpieza y análisis de los datos. Editorial UOC. * Megan Squire (2015).
Clean Data. Packt Publishing Ltd. * Jiawei Han, Micheine Kamber, Jian
Pei (2012). Data mining: concepts and techniques. Morgan Kaufmann. *
Jason W. Osborne (2010). Data Cleaning Basics: Best Practices in Dealing
with Extreme Scores. Newborn and Infant Nursing Reviews; 10 (1):
pp.~1527-3369. \emph{Peter Dalgaard (2008). Introductory statistics with
R. Springer Science \& Business Media. Wes McKinney (2012). Python for
Data Analysis. O'Reilley Media, Inc. } Tutorial de Github
\url{https://guides.github.com/activities/hello-world}.

\end{document}
